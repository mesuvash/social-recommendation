\begin{quote}
Content recommendation in social networks poses the complex problem of
learning user preferences from a rich and complex set of interactions
(e.g., likes, comments and tags for posts, photos and videos) and
activities (e.g., favorites, group memberships, interests).  While
many social collaborative filtering approaches learn from aggregate statistics over this
social information, we propose a different approach: we first define
social affinity groups (SAGs) of a target user by analysing their
fine-grained interactions (e.g., users who have been tagged in the
target user's video) and activities (e.g., users who have joined the
same special interest group that the target user has joined).  Then we
learn which SAGs are most predictive of the target user's preferences
in a method we term social affinity filtering (SAF).  We apply SAF to
data consisting of the preferences of 100+ Facebook users and their
complete interactions with 38,000+ friends collected over a four month
period.  Our analysis demonstrates that SAF yields higher accuracy
than competing social collaborative filtering approaches in the literature and that not all
interactions and activities are equally predictive: among many
insights, we show certain user-to-user interactions are more
informative than others (tagging is often more informative than
commenting, video interactions are more informative than wall post
interactions) and we analyse trends in the relationship between the
size of activity-based SAGs and informativeness (small groups can be
highly informative while large groups are rarely informative).
\end{quote}
