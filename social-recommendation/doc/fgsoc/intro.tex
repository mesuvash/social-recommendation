%!TEX root = document.tex

\label{sec:introduction}

% motivate socical recommendation 
Online social networks such as Facebook record a rich set of user
preferences (likes of links, posts, photos, videos), user traits,
interactions and activities (conversation streams, tagging, group
memberships, interests, personal history and demographic data).  
This presents myriad new
dimensions to the recommendation problem by making available a rich
labeled graph structure of social content from which user preferences
can be learned and new recommendations can be made.  

Existing recommendation methods for social networks aggregate this
rich social information into a simple measure of user-to-user interaction 
that can be leveraged to model user homophily via social
regularization~\cite{lla,socinf,sr,rrmf, Noel2012NOF}, a trust
ensemble~\cite{ste}, or a low-rank factorization of the social
interactions matrix~\cite{sorec}.  But in aggregating all of these
interactions and common activities into a single strength of
interaction, we ask whether important preference information has been
discarded?  Indeed, the point of departure for this work is the
hypothesis that different fine-grained interactions (e.g. commenting
on a wall vs. getting tagged in a video) and activities (e.g., a
university alumni group vs fans of a TV series) \emph{do} represent different
preferential {\em affinities} between users, and moreover that effective
{\em filtering} of this information will lead to better results in
social recommendation.

%% No longer fits cleanly in the intro, should be moved into related
%% work is not already mentioned there.  -SPS
\eat{
In the context of recent work on social
recommendation~\cite{sorec,ste,lla} and information diffusion
~\cite{Goel2012structure,Romero2011hashtag,Bakshy2012chamber}, it is
important to know which of these interactions or common traits are
actually reflective of common preferences.}

To quantitatively validate our hypotheses and evaluate the
informativeness of different fine-grained features for social
recommendation, we have built a Facebook App to collect detailed user
interaction and activity history available through the Facebook Graph
API along with preferences solicited from the App regarding recommendations.
Specifically, our App recommends three daily links to each
App user collected from the timeline of other users (both friends and
non-friends) and we record users' explicit likes and dislikes of these
recommended links.  Given this data, we define \emph{social affinity
groups (SAGs)} of a target user by analysing their fine-grained
interactions (e.g., users who have been tagged in the target user's
video) and activities (e.g., users who have joined the same special
interest group that the target user has joined).  Given these SAGS, we
(a) learn to predict whether a user will like an item based on members
of other SAGs who have also liked the item using a novel
recommendation method we call {\em social affinity filtering (SAF)},
and (2) analyse the relative informativeness of different SAGs based
on various properties that we elaborate on next.

In our 4-month interaction trace, we were able to collect data for a set of 
Facebook app users and their full interactions with 38,000+ friends along with 22
distinct types of interaction and users activity for 3000+ groups, 4000+ favourites, and 10,000+ pages. 
In subsequent sections that outline our experimental methodology and results in detail, 
we have made the following important observations:
\begin{itemize}
\item We found that SAF significantly 
outperforms numerous state-of-the-art collaborative filtering and social recommender 
systems, by up to 6\% in accuracy -- in short, fine-grained 
interactions are extremely informative, bringing into question the efficacy of 
previous social recommendation approaches that aggregate user-to-user interactions into 
a single value.
% Also, what about combining all features?  Not enough data?  -SPS
% Probably also much faster -- should we show a table of train and test times?  -SPS
\item We also found that groups, pages, and favourites make for more informative
SAGs than those defined by user-to-user interactions -- likely because the former can be
applied to SAGs over the entire Facebook population 
rather than just a user's friends (where the data is considerably limited).
\item Among the interactions, we found that those on videos are more predictive than those on other content types (photos, post, link), and that outgoing interactions (performed by the ego) 
are more predictive than incoming ones (performed by friends on the ego's timeline).
% Below: not sure I quite understand ``persistent'' and ``temporally synchronized'' here... 
% are there better terms or can they be further (briefly) explained?
\item Among {\em groups}, {\em pages} and {\em favourites}, we found that the most  
predictive features have smaller membership size, and that favourite features corresponding to 
persistent activities (such as TV, books, activities) are more predictive than generic or 
temporally synchronized activities (such as interests, sports). 
\end{itemize}
As detailed in the subsequent sections, these findings 
not only demonstrate the power of leveraging fine-grained
interaction and activity features but they also suggest which features are most
important to collect when building SAF-based recommenders.


\eat{
To better understand these subtleties and to understand what
social interactions and user traits reflect common preferences on
Facebook, we proceed in the following sections to describe our data,
our experimental methodology, and various analyses according to our
methodology that shed light on the above questions. 
On one hand our observations confirm certain observations made previous 
on different networks, such as the diminishing returns of repeated exposures, 
on the other we also see a few new clues such as 
that very specific types of outgoing interactions are more predictive 
than other interactions. 
We then conclude with a summary of the key novel observations arising
from this study.
}

\yum

