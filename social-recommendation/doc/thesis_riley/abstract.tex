%%
%% Template abstract.tex
%%

\chapter*{Abstract}
\label{cha:abstract}
\addcontentsline{toc}{chapter}{Abstract}

Social networks such as Facebook allow users to create a rich and verbose profile composed of both user specific interactions (such as 
comment and message passing, tags, likes, etc) and user preferences (such as favourite movies and music, group memberships, page likes, etc). 
These distinct components of a users profile can be leveraged to gauge their affinity with certain links and ultimately predict their
explicit like preferences.

The objective of this thesis is to decipher which of these aforementioned affinity measures are truly predictive of a users like preferences.

The success of our predictions are evaluated using the machine learning algorithms of  
\emph{Naive Bayes}, \emph{Logistic Regression} and \emph{Support Vector Machines}, results are compared to previous 
work using the state of the art social collaborative filtering technique of \emph{Social Matchbox} as a baseline. 
The data is sourced from a set of over 100 Facebook users and their interactions with over 39,000+ friends during a four month period.

Our analysis has shown that user interactions in themselves are not highly predictive of user likes, however user preferences are. 
This increasingly predictive trend continues as we limit user exposure (ensuring some $k$ number of users $u$ have liked a link $v$) over links. 
We conclude by analysing a combination of the most predictive user preference affinity measures, offer a summary of our work to date and propose 
recommendations for further research in this area.

%%% Local Variables: 
%%% mode: latex
%%% TeX-master: "thesis"
%%% End: 
