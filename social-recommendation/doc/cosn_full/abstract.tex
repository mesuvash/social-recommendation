% What do your interactions and activities say about your preferences?
% You are your Facebook page likes!
%
% Some interactions or activities are much more predictive than others
% Hence learning which ones are useful is crucial is of paramount importance
% The most useful features are the ones an App should request -- important
%   since 

Content recommendation in social networks poses the complex problem of
learning user preferences from a rich and complex set of interactions
(e.g., likes, comments and tags for posts, photos and videos) and
activities (e.g., favourites, group memberships, interests).  While
many social collaborative filtering approaches learn from aggregate
statistics over this social information, we show that only a small
subset of user interactions and activities are actually useful for
social recommendation, hence learning \emph{which} of these are most
informative is of critical importance.  To this end, we define a novel
social collaborative filtering approach termed social affinity
filtering (SAF).  On a preference dataset of Facebook users and their
interactions with 37,000+ friends collected over a four month period,
SAF learns which fine-grained interactions and activities are
informative and \emph{outperforms} state-of-the-art (social)
collaborative filtering methods by over 6\% in prediction accuracy.
In addition, we carry out an analysis of various fine-grained social
features and show among other insights that interactions on video
content are more informative than other interaction modalities (e.g.,
photos), the most informative activity groups tend to have small
memberships, and features corresponding to ``long-tailed'' content
(e.g., music and books) can be much more predictive than those with
fewer choices (e.g., interests and sports).  In summary, this work
demonstrates the substantial predictive power of fine-grained social
interaction and activity features and the novel method of SAF to
leverage them for state-of-the-art social recommender systems.

%while common music
%preferences are less predictive on average than other favourite
%categories, some obscure music preferences are among the most
%informative features of all favourites.  

