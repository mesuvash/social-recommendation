%!TEX root = document.tex

We proposed Social Affinity Filtering (SAF) as a novel method for social
recommendation that analyses a user's fine-grained interactions and
activities to learn which subset of social groups have the highest affinity with
a user's preferences.  We evaluated SAF on a dataset collected from a
Facebook App, showing that SAF yields 6\% absolute improvement in
accuracy over state-of-the-art (social) recommendation engines just
using knowledge of users' page likes.  This is an important result
given that SAF is built on standard supervised linear classification
techniques (which support strong training guarantees and fast learning
algorithms) and would be fairly robust in an online setting 
in contrast to some more complex matrix factorization optimisation approaches
often proposed for (social) collaborative filtering.

In addition to the strong social recommendation accuracy improvements
offered by SAF, we quantified the relative importance of interaction
and activity groups for recommendation and we analysed what properties
made some social affinity groups more informative than others.  Among
many insights, our results show that video interactions are more
predictive than other modalities and outgoing interactions are more
predictive than incoming ones, but both can be superceded by the
number of preferences expressed by a group.  Furthermore, for
activities, we showed that smaller social groups are more predictive
than larger ones and long-tailed categories with many specialised
choices (dynamically increasing over time) tend to contain some of
the most predictive affinity groups.  Knowing what subset of features
are informative allows one to design an effective recommendation tool
that requires minimal permissions from a user --- a key property for
general user uptake.

Future directions of research can 
%investigate the nature of
%interactions by measuring the level of user engagement --- e.g. if
%videos are inherently more engaging, or 
examine the nature of social groups via additional metrics --- e.g.,
the social network within members of the group, or activity level of
the group.  Other work might explore the cold-start advantages of SAF, 
improved feature engineering to better incorporate
preference frequency, or even combinations of SAF with
orthogonal (social) collaborative filtering approaches like nearest
neighbor or even matrix factorization.

%% Latent projections of affinity groups!
%% Exploit quantity in features and interactions/nonlinearity
%% Cold-start

%% Explore beyond one link out... two links, three links... how?
%% Try latent factorization approach?
%% Taxonomic features?
