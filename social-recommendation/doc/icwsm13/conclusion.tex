%!TEX root = document.tex

We proposed Social Affinity Filtering as a new method for social recommendation.
We define social affinities as the fine-grained actions that user perform in an
online social network, such as commenting on videos, tagging for photos, 
membership in a particular group, or subscription to a Facebook page. 
We design social affinity features as a new way to capture two users' similarity 
in the social action space, and use it to predict a users' preference for a link
being recommended to them. 
We evaluate the proposed algorithm on a dataset collected from a Facebook App we built, 
SAF has showed 10\% relative improvement with respect to 
state-of-the-art social recommendation engine. 
Furthermore, we quantified the relative importance of interactions and activities
for recommendation. Results show that video and photo interactions are more predictive 
than other modalities, and outgoing interactions are more predictive than incoming, 
and that smaller social groups are more predictive than larger ones. 

Future directions of research can investigate the nature of interactions by measuring 
the level of user engagement (e.g. are videos inherently more engaging), 
or examine the nature of social groups via additional metrics (e.g. the social network within members of the group, or activity level of the group). 