%!TEX root = document.tex


We built a Facebook App\footnote{Name and link omitted
for anonymity.} and  collect detailed user interaction and activity history. 
Over 119 users installed the Facebook App during the evaluation
period.\footnote{In order to collect full interactions among App users
as required in our experimentation, our App requested to collect
information posted to \emph{friends'} timelines.  With such expressive
permissions concerning friend interactions, most potential users were
hesitant to install the App; hence, after an intensive one month user drive period
at our University, we were able to attract the pool of users used in the experiments}
From these core App users, the App has access to their detailed
Facebook profiles and their \emph{complete} interactions with a total of 38,259
friends.

Our App tracks app user's (and their friends') details and
interactions on Facebook. We distinguish four Facebook items from wall posts:
general posts (e.g.,status updates, activity updates such as new friends), 
links, photos and videos. Facebook allows commenting, liking and tagging on these items.
Furthermore, Facebook allows user to interact with entities
(groups, pages and favourites) via membership and liking. The App also
keep tracks of the information of user's group membership, page likes
and favourites. We summarise relevant basic statistics of the data in
Tables~\ref{tab:interactions}--\ref{tab:likeinfo}.
Table~\ref{tab:interactions} summarizes the number of records for each
item (row) and interaction (column)
combination. Table \ref{tab:interests} shows the group membership,
page like and favourites counts for users.

%We built a Facebook App\footnote{Name and link omitted
%for anonymity.} to collect detailed user
%interaction and activity history available through the Facebook Graph
%API to build the SAG information as defined in Sec~\ref{ssec:sag}
%along with user $u$'s preferences on recommended links $i$ solicited by the App on a daily basis
%to build the $\likes(u,i)$ relation required in Sec~\ref{ssec:SAfeature}.
%The data collection is performed with full permission from the user and in
%accordance with an approved Ethics Protocol\footnote{Link omitted for
%anonymity.}.

%Over 119 users installed the Facebook App during the evaluation
%period.\footnote{In order to collect full interactions among App users
%as required in our experimentation, our App requested to collect
%information posted to \emph{friends'} timelines.  With such expressive
%permissions concerning friend interactions, most potential users were
%hesitant to install the App; hence, after an intensive one month user drive period
%at our University, we were able to attract the pool of users used in the experiments.  The difficulty
%of collecting data with such expressive App permissions suggests the importance
%of identifying a small subset of social features and permissions required to 
%obtain them in order to encourage App adoption by a wide audience.}
%From these core App users, the App has access to their detailed
%Facebook profiles and their \emph{complete} interactions with a total of 38,259
%friends.

% Our App tracks app user's (and their friends') details and
% interactions on Facebook.  Interactions that occur through wall posts
% provide a rich variety of content and interaction data.  We
% distinguish four Facebook items from wall posts: general posts (e.g.,
% status updates, activity updates such as new friends), links, photos
% and videos. Four main interactions on these items are permitted by
% Facebook: posting an item to a friend's wall, commenting, liking, and
% tagging. Furthermore, Facebook allows user to interact with entities
% (groups, pages and favourites) via membership and liking. The App also
% keep tracks of the information of user's group membership, page likes
% and favourites.  The App does not track deletions of these items,
% interactions (e.g., unlike) and memberships for performance reasons
% and we found very few deletions during an initial testing stage.  We
% summarise relevant basic statistics of the data in
% Tables~\ref{tab:interactions}--\ref{tab:likeinfo}.
% Table~\ref{tab:interactions} summarizes the number of records for each
% item (row) and interaction (column)
% combination. Table \ref{tab:interests} shows the group membership,
% page like and favourites counts for users.
%Table~\ref{tab:demographics} shows some demographics from %user profiles. 

Our app recommends three links collected from friends or non-friends to users every day, 
where the users may give their feedback on the links indicating whether they liked it or
disliked it. Throughout this paper, App user rated link data is used to evaluate the recommendation algorithms.
Table~\ref{tab:likeinfo} shows the statistics of the app user rating
for friend and non-friend recommendations.  We chose to only display
three links per day in order to avoid rank-bias with preferences.

%Crucially we note that all data used in the subsequent experiments
%is offline batch data that has been stored and analyzed after the four month
%data collection period.  
      							
%\begin{table}
%\centering
%\begin{tabular}{|>{\small}l|>{\small}r|>{\small}r|}
%\hline
%& \textbf{App Users} & \textbf{Ego network} \\
%& & \textbf{of App Users} \\
%\hline
%Users & 119 & 38,378 \\
%\hline
%Male & 85 & 20,840 \\
%\hline
%Female & 33 & 17,032 \\
%\hline
%\end{tabular}
%\caption{App user demographics}
%\label{tab:demographics}
%\end{table}

\begin{table}
\centering
\begin{tabular}{|>{\small}l|>{\small}r|>{\small}r|>{\small}r|}
\hline
\textbf{App Users} & \textbf{Tags} & \textbf{Comments} & \textbf{Likes} \\
\hline
\textbf{Post} & 7,711 & 22,388 & 15,999 \\
\hline
\textbf{Link}  & --- & 7,483 & 6,566 \\
\hline
\textbf{Photo} & 28,341 & 10,976 & 8,612 \\
\hline
\textbf{Video} & 2,525 & 1,970 & 843 \\
\hline
\hline
\textbf{Ego network} & \textbf{Tags} & \textbf{Comments} & \textbf{Likes} \\
\textbf{of App Users}  & & & \\
\hline
\textbf{Post} & 1,215,382 & 3,122,019 & 1,887,497 \\
\hline
\textbf{Link} & --- & 891,986 & 995,214 \\
\hline
\textbf{Photo} & 9,620,708 & 3,431,321 & 2,469,859 \\
\hline
\textbf{Video} & 904,604 & 486,677 & 332,619 \\
\hline
\end{tabular}
\caption{Statistics on user {\em Interactions}, grouped by item {\em Modality} and {\em Action type}.}
\label{tab:interactions}
\end{table}


\begin{table}[t!]
\centering
\begin{tabular}{|>{\small}l|>{\small}r|>{\small}r|}
\hline
& \textbf{App Users} & \textbf{Ego Network} \\
& & \textbf{of App Users} \\
\hline
Groups & 3,469 & 373,608 \\
\hline
Page Likes & 10,771 & 825,452 \\
\hline
Favourites & 4,284 & 892,820\\
\hline
\end{tabular}
\caption{Statistics on user {\em Actions}, counted for {\em groups, pages} and {\em favorites} over the App users and their ego network.}
\label{tab:interests}
\end{table}

\begin{table}[t!]
\centering
\begin{tabular}{|>{\small}l|>{\small}r|>{\small}r|}\hline
&\textbf{Friend}  & \textbf{Non-Friend} \\
&\textbf{recommendation}  & \textbf{recommendation} \\
\hline
Like& 1392 & 1127 \\
\hline
Dislike& 895 & 2111\\
\hline
\end{tabular}
\caption{Dataset breakdown of prediction target $like(u,i)$, by the source of the link (friend/non-friend) and value (like/dislike).}
\label{tab:likeinfo}
\end{table}


\eat{ %% suvash's original table
\begin{table}
\centering
\begin{tabular}{|>{\small}l|>{\small}r|>{\small}r|>{\small}r|>{\small}r|}
\hline
\textbf{App Users} & \textbf{Posts} & \textbf{Tags} & \textbf{Comments} & \textbf{Likes} \\
\hline
\textbf{Wall} & 36,359 & 7,711 & 22,388 & 15,999 \\
\hline
\textbf{Link} & 5,304 & --- & 7,483 & 6,566 \\
\hline
\textbf{Photo} & 4,933 & 28,341 & 10,976 & 8,612 \\
\hline
\textbf{Video} & 245 & 2,525 & 1,970 & 843 \\
\hline
\hline
\textbf{App Users} & \textbf{Posts} & \textbf{Tags} & \textbf{Comments} & \textbf{Likes} \\
\textbf{and Friends} & & & & \\
\hline
\textbf{Wall} & 4,301,306 & 1,215,382 & 3,122,019 & 1,887,497 \\
\hline
\textbf{Link} & 678,612 & --- & 891,986 & 995,214 \\
\hline
\textbf{Photo} & 1,268,816 & 9,620,708 & 3,431,321 & 2,469,859 \\
\hline
\textbf{Video} & 59,244 & 904,604 & 486,677 & 332,619 \\
\hline
\end{tabular}
\caption{The number of interactions Tables. Rows are type of Facebook item and columns are type of Facebook interaction.}
\label{tab:interactions}
\end{table}
}
