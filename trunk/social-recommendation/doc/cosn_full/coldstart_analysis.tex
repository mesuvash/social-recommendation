%#suvash#
Collaborative filtering algorithms suffer from the user cold-start problem,
where no historical information about user is available. One of the advantage of SAF is that 
the learning is indepenant of individual user. Hence, unlike standard (social) collaborative
filtering , SCF is robust to user cold-start problem.  For user cold-start analysis, we train
two SAF models by including and holding out test users from the training set and then evaluate
the performance of predictor on test set. Table ~\ref{tab:coldstart} shows that the accuracy 
of the SAF predictor for cold-start and non-cold start case is comparable.  

\begin{table}[t!]
\centering
\begin{tabular}{|>{\small}l|>{\small}r|>{\small}r|}
\hline
& \textbf{Accuracy}&\\
\hline
\textbf{Predictor}& \textbf{Cold-Start} & \textbf{Non Cold-Start}\\
\hline
\textbf{Constant} & 0.526  +/-  0.063 & 0.526  +/-  0.063 \\
\hline
\textbf{LR-ISAF} & 0.613 +/- 0.025 & 0.633  +/-  0.045 \\
\hline
\textbf{LR-ASAF(Groups)} & 0.625  +/-  0.025 & 0.651  +/-  0.023 \\
\hline
\textbf{LR-ASAF(Pages)} & 0.601  +/-  0.067 & 0.655  +/-  0.036 \\
\hline
\textbf{LR-ASAF(Favourites)} & 0.616  +/-  0.049 & 0.653  +/-  0.042\\
\hline
\end{tabular}
\caption{Performance evaluation of SAF for user cold-start}
\label{tab:coldstart}
\end{table}