%!TEX root = document.tex

We built a Facebook App to collect detailed user interaction and
activity history available through the Facebook Graph API.  The
purpose of the App was two-fold: (1) to build the SAG information as
defined in Sec~\ref{ssec:sag}, and (2) to collect a user $u$'s preferences on
recommended links $i$ solicited by the App on a daily basis to build
the $\likes(u,i)$ relation required in Sec~\ref{ssec:SAfeature}.  The
data collection is performed with full permission from the user and in
accordance with an approved Ethics Protocol\footnote{Ethics approval 
\#2011/142 from the Australian National University.}

In order to collect full interactions among App users as required in
our experimentation, our App requested to collect information posted
to \emph{friends'} timelines.  With such expressive permissions
concerning friend interactions, most potential users were hesitant to
install the App; after an intensive one month user drive period at our
University, we were able to attract 119 App users.  From these core
App users, the App has access to their detailed Facebook profiles and
their \emph{complete} interactions with a total of 37,872 friends.
The difficulty of collecting data with such expressive App permissions
suggests the importance of identifying informative SAGs for social
recommendation and the \emph{minimal} set of permissions required to
populate these SAGs to increase user uptake of a social recommendation
App.

Our App tracks user's (and their friends') details and interactions on
Facebook.  Interactions that occur via wall posts provide a rich
variety of content and interaction data.  We distinguish
four \emph{modalities} for Facebook timeline post content defined previously:
general wall posts (e.g., status updates, activity updates such as new
friends), links, photos and videos. Four main \emph{actions} on these
items are permitted by Facebook: posting an item to a friend's wall,
commenting, liking, and tagging. Furthermore, Facebook allows users to
engage in public \emph{activities} (groups, pages and favourites) via membership
and liking. In addition to interactions, the App also tracks users'
group membership, page likes, and favourites.
%The App does not track deletions of these items,
%interactions (e.g., unlike) and memberships for performance reasons
%and we found very few deletions during an initial testing stage.  
We summarise relevant basic statistics of the data in
Tables~\ref{tab:interactions}--\ref{tab:likeinfo}.
Table~\ref{tab:demographics} shows some demographics from user
profiles.  Table~\ref{tab:interactions} summarises the number of
records for each item \emph{modality} (row) and \emph{action} (column)
combination. Table \ref{tab:interests} shows the group membership,
page like and favourite counts for users.

Our App recommends three links to users every day, where the users may
give their feedback on the links indicating whether they liked or
disliked it. Users are recommended links that are collected
from \emph{both} friends' and non-friends' timelines.  We chose to 
display only three links per day in order to avoid rank-bias with
preferences; each link could be independently rated.
% and in
%general we observed that a user rated all three links on a given day
%that they viewed their recommendations.  
Table~\ref{tab:likeinfo}
shows the statistics of the App user rating for links taken from the
timelines of friends and non-friends.  

Lastly, we remark that all data used in the experiments is offline
batch data that has been stored and analysed \emph{after} the four
month data collection period.
      							
\begin{table}[t!]
\centering
\begin{tabular}{|>{\small}l|>{\small}r|>{\small}r|}
\hline
& \textbf{App Users} & \textbf{Ego network} \\
& & \textbf{of App Users} \\
\hline
Total & 119 & 37,872 \\
\hline \hline
Male & 85 & 20,840 \\
\hline
Female & 34 & 17,032 \\
\hline
\end{tabular}
\caption{App user demographics.  The \emph{ego network} is the friend
network of the App users.}
\label{tab:demographics}
\end{table}

\begin{table}[t!]
\centering
\begin{tabular}{|>{\small}l|>{\small}r|>{\small}r|>{\small}r|}
\hline
\textbf{App Users} & \textbf{Tags} & \textbf{Comments} & \textbf{Likes} \\
\hline
\textbf{Post} & 7,711 & 22,388 & 15,999 \\
\hline
\textbf{Link}  & --- & 7,483 & 6,566 \\
\hline
\textbf{Photo} & 28,341 & 10,976 & 8,612 \\
\hline
\textbf{Video} & 2,525 & 1,970 & 843 \\
\hline
\hline
\textbf{Ego network} & \textbf{Tags} & \textbf{Comments} & \textbf{Likes} \\
\textbf{of App Users}  & & & \\
\hline
\textbf{Post} & 1,215,382 & 3,122,019 & 1,887,497 \\
\hline
\textbf{Link} & --- & 891,986 & 995,214 \\
\hline
\textbf{Photo} & 9,620,708 & 3,431,321 & 2,469,859 \\
\hline
\textbf{Video} & 904,604 & 486,677 & 332,619 \\
\hline
\end{tabular}
\caption{Statistics on user {\em interactions}, grouped by item {\em modality} and {\em action type}.}
\label{tab:interactions}
\end{table}

