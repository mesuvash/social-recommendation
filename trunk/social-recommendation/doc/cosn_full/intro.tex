%!TEX root = document.tex

\label{sec:introduction}

% motivate socical recommendation 
Online social networks such as Facebook record a rich set of user
preferences (likes of links, posts, photos, videos), user traits,
interactions and activities (conversation streams, tagging, group
memberships, interests, personal history and demographic data).  This
presents myriad new dimensions to the recommendation problem by making
available a rich labeled graph structure of social interactions and
content from which user preferences can be learned and new
recommendations can be made.

Existing recommendation methods for social networks aggregate this
rich social information into a simple measure of user-to-user interaction 
that can be leveraged to model user homophily via social
regularization~\cite{lla,socinf,sr,rrmf, Noel2012NOF}, a trust
ensemble~\cite{ste}, or a low-rank factorization of the social
interactions matrix~\cite{sorec}.  But in aggregating all of these
interactions and common activities into a single strength of
interaction, we 
%show effectively that the signal present in this 
%information gets drowned out in the noise --- some fine-grained
%interactions and activities are highly informative, but most are not.
ask whether important preference information has been discarded?
Indeed, the point of departure for this work is the hypothesis that
different fine-grained interactions (e.g. commenting on a wall or
getting tagged in a video) and activities (e.g., being a member of a
university alumni group or a fan of a TV series) \emph{do} represent
different preferential {\em affinities} between users, and moreover
that effective {\em filtering} of this information (i.e., learning
which of these myriad fine-grained interactions and activities are 
informative) will lead to improved accuracy in social recommendation.

%% No longer fits cleanly in the intro, should be moved into related
%% work is not already mentioned there.  -SPS
\eat{
In the context of recent work on social
recommendation~\cite{sorec,ste,lla} and information diffusion
~\cite{Goel2012structure,Romero2011hashtag,Bakshy2012chamber}, it is
important to know which of these interactions or common traits are
actually reflective of common preferences.}

To quantitatively validate our hypotheses and evaluate the
informativeness of different fine-grained features for social
recommendation, we have built a Facebook App to collect detailed user
interaction and activity history available through the Facebook Graph
API along with user preferences solicited by the App on a daily basis.
Specifically, each day our App recommends three links to each App user
that are collected from the timeline of other users (both friends and
non-friends) and we record users' explicit likes and dislikes of these
recommended links.  Given this data, we define \emph{social affinity
groups (SAGs)} of a target user (ego) as sets of surrogate users
(alters) serving as proxies for an ego's preferences; these SAGs are
defined according to fine-grained
\emph{interactions} (e.g., users who have been tagged in the target user's
video) and \emph{activities} (e.g., users who have liked the same
political party page that the target user has liked).  Given a set of
SAGS for a user, (1) we define a novel recommendation method called
{\em social affinity filtering (SAF)}, where we learn to predict
whether a user will like an item based on the surrogate item
preferences of members of each SAG, and (2) we analyse the relative
informativeness of SAGs across a variety of dimensions such as type,
exposure to friends' preferences, and activity membership size.

In the four months that our App was active, we collected data for a
set of Facebook app users and their full interactions with 38,000+
friends along with 22 distinct types of interaction and users activity
for 3000+ groups, 4000+ favourites, and 10,000+ pages.  In subsequent
sections that outline our experimental methodology and results in
detail, we make the following critical observations:
\begin{itemize}
\item We found that SAF significantly 
outperforms numerous state-of-the-art collaborative filtering and social recommender 
systems, by up to 6\% in accuracy using just \emph{page} (like) features.
Because we have observed that the reluctance of a user to install an App increases with the number
of permissions requested, this suggests a state-of-the-art social recommendation App 
need only request permissions for a user's \emph{page} likes in order to achieve
state-of-the-art recommendation accuracy.
% -- in short, these fine-grained features can be very informative.
% Also, what about combining all features?  Not enough data?  -SPS
% Probably also much faster -- should we show a table of train and test times?  -SPS
\item We found that \emph{groups}, \emph{pages}, and \emph{favourites} make for more informative
SAGs than those defined by user-to-user interactions.  This is likely because the former can be
applied to SAGs over the entire Facebook population 
rather than just a user's friends (where the available preference data is considerably more sparse).
\item Among the \emph{interactions}, we found that those on videos are more predictive than those on other content types (photos, post, link), and that outgoing interactions (performed by the ego on the alter's timeline) 
are more predictive than incoming ones (performed by alters on the
ego's timeline), although the level of exposure of an ego to an
alter's preferences is more important than the directionality of the
interaction with the alters.
% Below: not sure I quite understand ``persistent'' and ``temporally synchronized'' here... 
% are there better terms or can they be further (briefly) explained?  -SPS
\item %{\bf TODO: see Latex comments.} 
Among {\em groups}, {\em pages} and {\em favourites}, we found that only
a small subset of the features were actually informative, bringing into question the efficacy of 
previous social recommendation approaches that aggregate social information between
two users into a single numerical value.  We found that the most  
predictive activity SAGs tend to have small memberships.  We also found features
corresponding to ``long-tailed'' content (such as music and books)
can be much more predictive than those with fewer choices 
(e.g. interests or sports). 
 %to generic (such as interests) or activities that require 
 %simultaneous participation (such as sports) are less predictive of 
 %user interest than topics that are ongoing or persistent in time (such as TV, books).
\end{itemize}
As detailed in the subsequent sections, these findings not only
demonstrate the power of leveraging fine-grained interaction and
activity features but they also suggest what social information can 
be most informative to collect when building state-of-the-art
SAF-based social recommenders.
%  This
%latter point is quite important since as already noted, there is
%a trade-off between the number of permissions a Facebook App requests
%and user uptake.
%
%we note later -- the more
%permissions an App requests, the less likely a user is to grant
%permissions to the App, so choosing permissions (i.e., social
%features) well is crucial for achieving good recommendations with
%minimal intrusion into a user's privacy.

\eat{
To better understand these subtleties and to understand what
social interactions and user traits reflect common preferences on
Facebook, we proceed in the following sections to describe our data,
our experimental methodology, and various analyses according to our
methodology that shed light on the above questions. 
On one hand our observations confirm certain observations made previous 
on different networks, such as the diminishing returns of repeated exposures, 
on the other we also see a few new clues such as 
that very specific types of outgoing interactions are more predictive 
than other interactions. 
We then conclude with a summary of the key novel observations arising
from this study.
}

\yum

