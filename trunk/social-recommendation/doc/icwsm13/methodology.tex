
% interactions
\cite{Panigrahy2012ubr}
\cite{Goel2012structure}
\cite{Wilson2012BSG}

The major objective of this paper is to evaluate the effectiveness of \textit{Social Affinity Filtering(SAF)} and fine grained 
analysis of the informativeness of Interactions and Activities.We divide our recommendation algorithm into two categories based 
on Interactions and Activities of target user's \textit{ Social Affinity Groups(SAG)} with target item, 
namely \textit{Interaction  Social Affinity Groups(ISAG)} and \textit{Activity Social Affinity Groups(ASAG)}.

\begin{itemize}
  \item \textbf{Interaction  Social Affinity Groups}: We define the set of interaction affinity classes based on 
  Interaction modality, action and direction as
  \begin{quote}
  \begin{math}
  	\textit{Interaction Affinity Classes} = \{link, post, photo, video\} \times \{like, tag, comment\} \times \{incoming, outgoing\}
  \end{math}
  \end{quote}
  %Additionally, we add friendship interaction affinity class. \\
  We define 
  \begin{quote}
  \textit{ISAG(u, i)} : set of the users who have interaction $i$ with user $u$.
  \end{quote}
   For example,
   \begin{quote}
   
   \textit{ISAG(u, link-like-incoming)} : set of all users who have liked link posted by user $u$. \\
   \textit{ISAG(u, link-like-outgoing)} : set of all users whose link is liked by user $u$. \\
   \end{quote}
\item \textbf{Activity Social Affinity Groups}: We define activity affinity groups based on group membership, page likes and user favourites.
	\begin{quote}
	\textit{ASAG(u, i)} : set of the users who share the entity $i$ (group, page, favourite) with user $u$.   
	\end{quote}
\end{itemize}
Finally, we define
\begin{quote}
\begin{math}
likes(u,i) =  \begin{Bmatrix}
	  True & \text{if user $u$ likes item $i$}\\
	  False & \text{otherwise}
	  \end{Bmatrix}
\end{math}
\end{quote}
\subsection{Social Affinity Features}
\begin{itemize}
  \item \textbf{Interaction Social Affinity Features} : We define Interaction affinity features for target user $u$ and item $i$ for ISAG's classes 
  $ \langle X_{1},X_{2}\ldots,X_{k}\rangle$ as
  \begin{quote}
  \begin{math}
   X_{k,u,i} = \begin{Bmatrix}
   		True & if\ \exists v\in ISAG(u,k) \wedge likes(v,i)\\ \\
   		False & otherwise
   \end{Bmatrix}
  \end{math}
  \end{quote}
  Additionally we add friend feature which encodes whether the target item $i$ is liked by friend or not.
  \item \textbf{Activity Social Affinity Features} : We define activity affinity features for target user $u$ and item $i$   \
  $ \langle X_{1},X_{2}\ldots,X_{k}\rangle$ as\\ \\
  \begin{math}
   X_{k,u,i} = \begin{Bmatrix}
   		True & if\ \exists\ v\in \ ASAG(u,k) \wedge likes(v,i)\\ \\
   		False & otherwise
   \end{Bmatrix}
  \end{math}
	In our analysis we use only those features(groups, pages and favourites) that is joined/liked by atleast one of our app user.
\end{itemize}

We train naive bayes, Logistic Regression(LR) and Support Vector Machine(SVM) model with affinity features.
Logistic Regression and SVM algorithm was implemented using \textit{LIBLINER} \cite{liblinear} package. 
We define Constant predictor as baseline predictor. Constant predictor predicts the most common outcome in our datase ie disliket.
We compare the performance of Social Affinity Filtering with the state of the art social collaborative filtering technique 
Social MatchBox(SMB)\cite{SMB}.

In real world social networks activity features grows very quickly as number of user in social network increases. This motivates the fine grained
analysis of informativeness of social affinity features. Furthermore, fine grained analysis of interactions helps to understand the nature of user-user 
interactions and its predictiveness in greater detail. Hence, for the analysis of activities and interactions we rank the features using
\textit{Conditional Entropy}. 
\textit{Conditional Entropy} is defined as
\begin{quote}
\begin{math}
H(Y|X=True) = \\-\sum_{y\in{(like,dislike)}} p(y|X$=$true)$ $log( p(y|X$=$true))
\end{math}
\end{quote}

With the data and methodology now defined including all dimensions of our analysis, we now proceed to an in-depth discussion of our findings.





 




