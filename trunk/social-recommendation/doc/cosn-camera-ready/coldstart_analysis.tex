%#suvash#

Many collaborative filtering algorithms (e.g., NN and MF) suffer from
the user \emph{cold-start} problem, i.e., when no preference data is
available for a user, these algorithms cannot perform better than the
constant (most likely class) predictor since they have no way of
generalising to a new unseen user.  Since SAF trains a single model
for all users and does \emph{not} require a user's preferences in
order to recommend for them, SAF can be used in a \emph{cold-start}
setting to recommend for users \emph{without} expressed item
preferences as long as those users have interactions or shared
activities with users who have expressed item preferences.

To quantitatively evaluate the cold-start performance of SAF, we run
10-fold cross validation with specially constructed folds.
For \textit{cold-start} evaluation, in each fold, we hold out a random
10\% subset of the users for testing and train on the remaining 90\%
of users.  In the test set, we further hold out 30\% of the test user
data.  In the \textit{non cold-start} evaluation, we test on the same
data as in the \textit{cold-start} evaluation, but we add in the 30\%
of held-out test user data to the \textit{cold-start} training set
thus allowing the \textit{non cold-start} setting to train on some of
the test user data.  In Fig~\ref{fig:coldstart}, we clearly see that
the accuracy\footnote{The slight decrease in accuracy for non
cold-start case compared to Fig~\ref{Fig1} is due to the decreased
amount of test user data present in the training set for this set of
experiments.}  of the SAF predictor for cold-start is significantly
better than the baseline Constant predictor.  Furthermore, the
accuracy of the cold-start predictor is actually comparable to the non
cold-start predictor, indicating that SAF exhibits strong
cold-start performance.

%Given the clearly demonstrated benefits of SAF, we now proceed in the
%next two subsections to analyse the two primary types of SAG features
%(interactions and activities) to better understand characteristics of
%both informative and uninformative SAGs in each context and the social
%phenomena that may be responsible for these characteristics.

%% SPS: Figure moved to saf_analysis.tex for positioning reasons.

%\begin{table}[t!]
%\centering
%\begin{tabular}{|>{\small}l|>{\small}r|>{\small}r|}
%\hline
%& \multicolumn{2}{|c|}{\textbf{Accuracy}}\\
%\hline
%\textbf{Predictor}& \textbf{Cold-Start} & \textbf{Non Cold-Start}\\
%\hline
%\textbf{Constant} & 0.526  +/-  0.063 & 0.526  +/-  0.063 \\
%\hline
%\textbf{LR-ISAF} & 0.613 +/- 0.025 & 0.633  +/-  0.045 \\
%\hline
%\textbf{LR-ASAF(Groups)} & 0.625  +/-  0.025 & 0.651  +/-  0.023 \\
%\hline
%\textbf{LR-ASAF(Pages)} & 0.601  +/-  0.067 & 0.655  +/-  0.036 \\
%\hline
%\textbf{LR-ASAF(Favourites)} & 0.616  +/-  0.049 & 0.653  +/-  0.042\\
%\hline
%\end{tabular}
%\caption{Comparision of performance of SAF for user cold-start and non cold-start case}
%\label{tab:coldstart}
%\end{table}
