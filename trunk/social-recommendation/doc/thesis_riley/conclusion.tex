%%
%% Template conclusion.tex
%%


\section{Summary}
\label{sec:summary}

In this thesis we have tested and compared different feature vectors across different exposures of size $k$. 
We have shown that \emph{User Interactions} in themselves are not predictive of user likes, however coupled with the 
likes exposure curve, they do show an improvement over our baselines.

We have also shown the interesting result that \emph{User Preferences} are predictive of user likes in the base case of $k = 0$ and 
this trend continues over the likes exposure curve as $k$ increases.

To answer the question initially proposed for this thesis, we have shown the feature vector which provides the highest predictiveness 
for user likes is the combination of the individually predictive feature vectors which is the novel insight provided by this thesis.

\section{Future Work}
\label{sec:fw}

Proposed future work can be summarised under the following points.

\begin{itemize}
\item \textbf{Increase size ranges}: Given our maximum test sizes for \emph{Groups} and \emph{Pages} of $1000$ this size could be 
increased to find the optimal testing size for each classifier
\item \textbf{Ranges tests}: \emph{Groups} and \emph{Pages} were tested based on their popularity, a more useful comparison could be to 
test based on some other condition, such as \emph{Group} or \emph{Page} size.
\item \textbf{Individual Traits analysis}: During our traits analysis the feature vector was set to $1$ if the user and any user in 
the set of alters were part of the same \emph{Traits} group, it would be beneficial to do an individual analysis on each component of 
the \emph{Traits} data to find which parts of \emph{Traits} are most predictive (similar analysis as done in \emph{Groups}).
\item \textbf{Passive likes}: Given the Facebook model of allowing users to like but not dislike data, explicit dislike data can not be gleaned
from Facebook, which is hence why the LinkR active likes data was used for this evaluation. An approach could be developed which can predict 
whether a user will have seen an item (online timestamps, recent interactions with user) and can infer that if the user did not like the item then
they disliked it.
\item \textbf{Cold start}: Leaving out some subset of users when training our models, but including them during testing.
\item \textbf{General user set}: Such as the study done by ~\cite{jugand} which comprised of the entire active social network of 721 million users 
as of May 2011, applying these methods to a data set which is more indicitive of the general Facebook user population would result in more generalsied 
results.
\item \textbf{Bayesian Model Averging}: Weighting the most successful machine learning models under different feature sets and exposure curves 
to simulate a new combined classifier.
\end{itemize}

%%% Local Variables: 
%%% mode: latex
%%% TeX-master: "thesis"
%%% End: 
