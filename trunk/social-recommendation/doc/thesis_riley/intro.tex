%%
%% Template intro.tex
%%

\chapter{Introduction}
\label{cha:intro}

\section{Social Networks}
\label{sec:sn}

The social network central for this study is Facebook. Once registering, Facebook users have the option of setting up a personalised profile, they can then establish
 themselves as friends of other users. Friends can interact via wall posts, conversations or by liking some facebook element.

Social networks such as Facebook provide a wide array of user preferences (link, tag, photo, video likes) in an array of interaction mediums and modalities 
(outgoing, incoming) as well as user specific information (gender, age, location, group memberships, favorite movies) and conversation content.

A problem with the Facebook paradigm in relation to this analysis is the requirement for assumed dislikes, if a user does not like some link can we 
imply the user does not like this link? Given the time period Facebook shows a link and the differing online times for Facebook users, this is generally 
a poor assumption. As such a Facebook app named LinkR was developed by NICTA which explicitly stores like and dislike data for users. This app will 
be discussed in the following section.

\section{Data Set}
\label{sec:data}

The LinkR Facebook app was used to collect information about users, their interactions and preferences. The data set contains information about app users as 
well as a sub-set of visible information about their friends. The app tracked and stored information for over 100 app users and their 39,000+ friends.

The four main interactions between users are posts (posting an element on a friends' wall), tags (being mentioned in a friends post or comment), 
 comments (written data on a post) and likes (clicking a like button if a user likes a post or comment). The table below outlines data collected during 
 app trials.

\begin{table}[!htbp]
\centering
	\begin{tabular}{|l|r|r|r|r|} % cols: (left, center, right)
		\hline
		\textbf{App Users} & \textbf{Posts} & \textbf{Tags} & \textbf{Comments} & \textbf{Likes}  \\ \hline
		\textbf{Wall} & 27,955 & 5,256 & 15,121 & 11,033 \\ \hline
		\textbf{Link} & 3,974 & - & 5,757 & 4,279 \\ \hline
		\textbf{Photo} & 4,147 & 22,633 & 8,677 & 5,938 \\ \hline
		\textbf{Video} & 211 & 2,105 & 1,687 & 710 \\ \hline
		 \hline
		\textbf{App Users and Friends} & \textbf{Posts} & \textbf{Tags} & \textbf{Comments} & \textbf{Likes}  \\ \hline
		\textbf{Wall} & 3,384,740 & 912,687 & 2,152,321 & 1,555,225 \\ \hline
		\textbf{Link} & 514,475 & - & 693,930 & 666,631 \\ \hline
		\textbf{Photo} & 1,098,679 & 8,407,822 & 2,978,635 & 1,960,138 \\ \hline
		\textbf{Video} & 56,241 & 858,054 & 463,401 & 308,763 \\ \hline
	\end{tabular}
	\caption{Total app user records}
	\label{tab:revpol}
\end{table}

\clearpage
\section{Goal}
\label{sec:goal}

The goal of this thesis is to discover which sub-set or combination of user interactions and/or user preferences will be the most predictive of user likes. 


%%% Local Variables: 
%%% mode: latex
%%% TeX-master: "thesis"
%%% End: 
