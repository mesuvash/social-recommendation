%%
%% Template intro.tex
%%

\chapter{Introduction}
\label{cha:intro}

An individuals social presence on the web is continually expanding, with the emergence of services such as Facebook, Myspace, LinkedIn, Twitter 
and Google+ what defines a user and their online social interactions (messages, posting, commenting, etc) and preferences 
(demographics, group memberships, likes, etc) is an ever expanding graph structure of verbose social content. Providing a myriad of expanding
social information and user driven content.

Our question becomes, how can we exploit this information to decipher which user interactions or preferences are most indicitive of user likes?
Can we leverage this information and generalise over a social population?

Internet is becoming a network of people

\section{Objectives}
\label{sec:objectives}

The primary objective of this thesis is to contrast and compare differing user feature sets. Using the
machine learning concepts of \emph{Naive Bayes}, \emph{Logistic Regression} and \emph{Support Vector Machines} compared with our 
appropriate baslines of \emph{Social Mathbox} and \emph{Constant Classifiers}.

Using these classification techniques we will compare both user interaction features 
\emph{interactions, incoming/outgoing messages} and user preference features \emph{demographics, traits, groups, pages}. To analyse the effect 
each feature set has on classification.

These algorithms will also be tested against a subset of the data, using a hold out technique to only test against items which has been liked
by some friend.

Finally, we will analyse the effect of combining successful user feature sets together and the results of combining successful models together 
using a \emph{Bayesian Model Averaging} approach.

\section{Contributions}
\label{sec:contributions}

In the preceding section, we outlined different user interaction and preference features. Our specific contributions made during this thesis show 


"Facebook users during the 2010 US congressional elections. The results show that the messages directly influenced political
self-expression, information seeking and real-world voting behaviour of millions of people. Furthermore, the messages not only 
influenced the users who received them but also the users’ friends, and friends of friends. The effect of social transmission on real-world
voting was greater than the direct effect of the messages themselves, and nearly all the transmission occurred between ‘close friends’
who were more likely to have a face-to-face relationship. These results suggest that strong ties are instrumental for spreading both
online and real-world behaviour in human social networks."
~\cite{nature}


discuss feed time for facebook

"Social influence can play a crucial role in a range of behav-
ioral phenomena, from the dissemination of information, to
the adoption of political opinions and technologies [23, 42],"
~\cite{grano}
~\cite{watts}


"has been shown that positive social annotations on search items adds perceived utiltiy to the worth of a result, particularly with close social
connections"
~\cite{pantel}






\begin{itemize}
\item Both \emph{interactions} and \emph{incoming/outgoing messages} are not more predictive then previously used \emph{Social Matchboxing} 
techniques.
\item Each user preference of \emph{demographics, traits, groups, pages} contributed to a better result then our baselines.
\item Combining user preferences with a hold out technique for user likes results in a sgnificant improvement from our baslines.
\item Combination of user preferences is better
\item Model combination can encapsulate these ideas
\end{itemize} 

Overall, we provide a methodology which improves upon previous work and offers a way to combine positively contributing aspects of different 
feature sets in our data.

\section{Outline}
\label{sec:outline}

The remaining chapters in this thesis are organised as follows:
\begin{itemize}
\item \textbf{Chapter 2}: We first outline appropriate background information for the reader. Including information pertaining to the source of 
the data set, notation used throughout this thesis, previous work in this area and our research approach and methodology
\item \textbf{Chapter 3}: In this chapter we discuss different feature sets for user interactions and the results from applying these feature sets 
in comparison with our baselines.
\item \textbf{Chapter 4}: Similarly as above, however we discuss user preferences
\item \textbf{Chapter 5}: In this chapter we discuss results from combining different feature sets and models
\item \textbf{Chapter 6}: Finally, we draw the work done throughout this thesis to a conclusion and offer avenues for future work in this area.
\end{itemize}

All chapters combined, this thesis represents a novel approach to which feature sets are predictive of user likes and offer an approach to 
combining positive components into a useful classifier.

%%% Local Variables: 
%%% mode: latex
%%% TeX-master: "thesis"
%%% End: 
