%%
%% Template intro.tex
%%

\chapter{Introduction}
\label{cha:intro}

The Internet is becoming a network of people, providing a myriad of expanding social information and user driven content. 
Social presence on the web is continually expanding. With the emergence of services such as Facebook, Myspace, LinkedIn, Twitter 
and Google+, what defines a user and their online \emph{user interactions} (such as comment and message passing, tags, likes) and \emph{user preferences} 
(such as favourite movies and music, group memberships, page likes) is an expanding graph of rich social content. 

From this premise, the ultimate question we wish to address in this thesis is: 
How can we leverage this user information to decipher which \emph{user interaction} or \emph{user preference} affinity features 
are most predictive of user likes? 

We address this question by comparing and contrasting these different potential affinity relationships in our data against appropriate 
baselines and ultimately offer a combination of features which offers our best solution to the question posed above.

In this chapter we will outline the objectives of this research, summarise the contributions made and provide an outline for the remaining
chapters.

\section{Objectives}
\label{sec:objectives}

One issue present in this Facebook paradigm is discovering whether a user doesn't like an item, a users Facebook feed is comprised of 
activity between their friends, content, groups, etc giving an enormous scope of potential feed items. Facebook will only show feed 
items for users who have recently interacted with using their \emph{Edge-Rank} ~\cite{edge} algorithm. 

While many Facebook users have a friend count which is close to the human real word limit, known as the Dunbar number 
~\cite{hill2003social}, the \emph{Edge-Rank} algorithm ensures user interactions are focused on a much smaller subset of their friends.
Additionally, given the rate of posting, these top feed items are only displayed for a short amount of time. Coupled 
with the fact that Facebook allows users to explicitly like an item, but not dislike it - distinguishing between what 
a user does and does not like becomes difficult.

The primary objective of this thesis is to compare and contrast differing potential affinity features across \emph{user interactions} and 
\emph{user preferences}. Using state of the art machine learning concepts of \emph{Naive Bayes} (NB), \emph{Logistic Regression} (LR) 
and \emph{Support Vector Machines} (SVM) compared with our appropriate baselines of \emph{Social Matchbox} (SMB) and \emph{Constant Classifiers}. 
With the primary aim of discovering which affinity features are most predictive or user likes.

Based on the insight that social influence can play a crucial role in a range of behavioural phenomena \cite{grano,watts} and that
positive social annotations on search items add perceived utility to the worth of a result \cite{pantel} we will 
also test using an exposure hold out technique, where data is only tested if some friend has already liked that item. Hence
the need to undertake an analysis of the effect of exposure on our affinity features.

Finally, we will assess and compare the effect of combining successful individual affinity features found during our analysis.

\section{Contributions}
\label{sec:contributions}

Our specific contributions made during this thesis show:

\begin{itemize}
\item Both \emph{interactions} and \emph{incoming \textbackslash outgoing messages} posted between users are not more predictive than 
previously used SMB techniques.
\item Each \emph{user preference} affinity of \emph{favourites} (such as favourite movies, music), \emph{group} memberships (such as Australian National University
and Students in Canberra) and \emph{page} likes (such as Google Chrome and The Simpsons) outperformed our baselines.
\item Combining both affinity types of \emph{user interactions} and \emph{user preferences} with an exposure limit resulted in a 
substantial improvement over previous techniques as the exposure increases.
\item Combination of the advantageous affinity features briefly mentioned above gives the best results in our analysis.
\end{itemize} 

Overall, we provide a methodology that improves upon previous work and offers an approach to combine predictive affinity features.

\section{Outline}
\label{sec:outline}

The remaining chapters in this thesis are organised as follows:
\begin{itemize}
\item \textbf{Chapter 2}: We first outline appropriate background information for the reader. Including information pertaining to the source of 
our data set, mathematical notation used throughout this thesis, previous work in this area and our research approach and methodology.
\item \textbf{Chapter 3}: In this chapter we discuss different affinity features for \emph{user interactions} and the results of applying these features to NB, SVM and LR 
in comparison with our baselines.
\item \textbf{Chapter 4}: A similar affinity feature analysis as above is applied, however the features we utilise are for \emph{user preferences}.
\item \textbf{Chapter 5}: In this chapter we discuss the effect of combining different affinity features based on results gained in the previous sections
and propose an ideal affinity feature hybrid.
\item \textbf{Chapter 6}: Finally, we draw the work done throughout this thesis to a conclusion and offer avenues for future work in this area.
\end{itemize}

With all chapters combined, this thesis represents a novel approach to exploiting and analysing \emph{user interactions} and \emph{user preferences} affinity relationships
to ascertain which features are most predictive of user likes and present an approach of combining these useful feature components into an effective 
classification paradigm.

%%% Local Variables: 
%%% mode: latex
%%% TeX-master: "thesis"
%%% End: 
