%%
%% Template intro.tex
%%

\chapter{Introduction}
\label{cha:intro}

The Internet is quickly becoming a network of people, providing a myriad of expanding social information and user driven content. 
This social presence on the web is continually expanding. With the emergence of services such as Facebook, Myspace, LinkedIn, Twitter 
and Google+, what defines a user and their online \emph{user interactions} (such as comment and message passing, tags, likes, etc) and \emph{user preferences} 
(such as favourite movies and music, group memberships, page likes, etc) is an expanding graph of rich social content. 

From this premise, the ultimate question we wish to address in this thesis becomes: 
How can we leverage this user information to decipher which \emph{user interaction} or \emph{user preference} affinity features 
are most predictive of user "likes"? 

We address this question by comparing and contrasting these varied potential affinity relationships in our data against appropriate 
baselines and ultimately offer a combination of features which proposes the best solution the question posed above.

In this chapter we will outline the objectives of this research, summarise our contributions and provide an outline for the remaining
chapters.

\section{Objectives}
\label{sec:objectives}

The primary objective of this thesis is to compare and contrast differing potential affinity features across both \emph{user interactions} and 
\emph{user preferences}. Using state of the art machine learning concepts of \emph{Naive Bayes} (NB), \emph{Logistic Regression} (LR) 
and \emph{Support Vector Machines} (SVM) compared with our appropriate baselines of \emph{Social Matchbox} (SMB) and \emph{Constant Classifiers} (Constant). 
With the primary aim of discovering which affinity features are most predictive or a users like preferences.

Based on the insight that social influence can play a crucial role in a range of behavioural phenomena \cite{grano,watts} and that
positive social annotations on search items add perceived utility to the worth of a result \cite{pantel} we will 
also test while limiting user exposure to a link. This exposure hold out technique involves ensuring some $k$ number of users $u$ have 
liked some link $v$.

Based on the results found during our \emph{user interaction} and \emph{user preference} affinity analysis we will also extract individual feature 
weights to analyse which explicit features are most predictive of a users like preferences 
(ie, for groups: which group sizes, group types, group localities, etc are most predictive).

Finally, we will assess and compare the effect of combining the individually predictive affinity features found during our analysis.

\section{Contributions}
\label{sec:contributions}

Most previous work has simply compressed all social information into a single "user similarity" metric, which does not leverage
the fact that very specific interactions or preferences can be very predictive. Our approach involves  
recommendation via affinity groups. This is a novel way to do social recommendation as it can leverage very fine-grained social features.

Specific contributions made during this thesis show:

\begin{itemize}
\item Both \emph{interactions} (tags, likes, etc) and \emph{messages} (incoming and outgoing messages posted between users) are not more predictive than 
previously applied SMB techniques.
\item The \emph{user preference} affinities of \emph{favourites} (favourite movies, music, etc), \emph{group} memberships (Australian National University, Students in Canberra, etc) 
and \emph{page} likes (Google Chrome, The Simpsons, etc) are the most predictive affinity features.
\item Comparing both affinity types of \emph{user interactions} and \emph{user preferences} against an exposure limit results in a 
substantial improvement over previous techniques as this exposure increases.
\item Individual \emph{groups} which were most predictive were highly localised with a medium user frequency, while the most predictive 
\emph{pages} were much more general and of a higher relative user frequency.
\item Combination of the most predictive individual affinity features outlined above present the most predictive results found during our analysis.
\end{itemize} 

Overall, we discover which user affinities are most predictive of a users like preferences (e.g., 
whether a user will "like" a certain news article or blog post), analyse the effect of user exposure across links, 
evaluate which explicit affinity features contribute the most weight during prediction and assess the predictive qualities of combining 
the individually most predictive affinities found during our analysis.

\section{Outline}
\label{sec:outline}

The remaining chapters in this thesis are organised as follows:
\begin{itemize}
\item \textbf{Chapter 2 (Background)}: We first outline appropriate background information for the reader. Including information pertaining to the source of 
our data set, mathematical notation used throughout this thesis, previous work in this area and our research approach and methodology.
\item \textbf{Chapter 3 (User Interactions)}: In this chapter we discuss different affinity features for \emph{user interactions} and the results of applying these features to NB, SVM and LR 
in comparison with our baselines.
\item \textbf{Chapter 4 (User Preferences)}: A similar affinity feature analysis as above is applied, however the features we utilise are for \emph{user preferences}.
\item \textbf{Chapter 5 (Feature Combination)}: In this chapter we discuss the effect of combining individually predictive affinity features based on results gained in the previous sections
and propose an ideal feature hybrid.
\item \textbf{Chapter 6 (Conclusion)}: Finally, we draw the work done throughout this thesis to a conclusion and offer avenues for future work in this area.
\end{itemize}

With all chapters combined, this thesis represents a novel approach to exploiting and analysing \emph{user interactions} and \emph{user preferences} affinity relationships
to ascertain which features are most predictive of user likes and present an approach of combining these useful individual feature components into an effective 
classification paradigm.

%%% Local Variables: 
%%% mode: latex
%%% TeX-master: "thesis"
%%% End: 
