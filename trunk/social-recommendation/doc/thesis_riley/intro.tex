%%
%% Template intro.tex
%%

\chapter{Introduction}
\label{cha:intro}

An individuals social presence on the web is continually expanding, with the emergence of services such as Facebook, Myspace, LinkedIn, Twitter 
and Google+ what defines a user and their online social interactions (messages, posting, commenting, etc) and preferences 
(demographics, group memberships, likes, etc) is an ever expanding graph structure of verbose social content. The Internet is becoming a network 
of people, providing a myriad of expanding social information and user driven content.

The ultimate question we wish to address in this thesis then becomes: 
How can we exploit this user information to decipher which \emph{User Interactions} or \emph{User Preferences} are most 
indicative of user likes? 

We address this question by comparing and contrasting different feature vectors in our data against appropriate baselines and 
ultimately offer a \emph{Positive Feature Combination} paradigm which represents improved results which are computationally faster and can more 
appropriately generalise over a population then previously explored methods.

\section{Objectives}
\label{sec:objectives}

The primary objective of this thesis is to contrast and compare differing user feature vectors across \emph{User Interactions} and 
\emph{User Preferences}. Using the machine learning concepts of NB, LR and SVM compared with our appropriate baselines of 
SMB and \emph{Constant Classifiers}. With the goal of discovering which features vectors are most predictive or user likes.

Based on the insight that social influence can play a crucial role in a range of behavioral phenomena \cite{grano,watts} we will 
also test using an exposure curve \cite{Romero2011hashtag} hold out technique, where data is only tested if some friend has 
already liked that item. ~\cite{pantel} has been shown that positive social annotations on search items adds perceived utility
to the worth of a result, implying that a previously liked item will be more predictive.

Finally, we will analyse the effect of combining successful user feature vectors together using a \emph{Positive Feature Combination} 
approach.

\section{Contributions}
\label{sec:contributions}

Our specific contributions made during this thesis show:

\begin{itemize}
\item Both \emph{Interactions} and \emph{Incoming\\Outgoing messages} are not more predictive then previously used SMB
techniques.
\item Each user preference of \emph{Traits}, \emph{Groups} and \emph{Pages} contributed to a better result then these previously used techniques.
\item Combining user preferences with an exposure curve for user likes results in a substantial improvement from previously used techniques.
\item \emph{Positive Feature Combination} of beneficial \emph{User Preferences} results in the most advantageous feature vector for this data.
\end{itemize} 

Overall, we provide a methodology which improves upon previous work and offers an approach to combine positively contributing aspects of different 
feature vectors in our data.

\section{Outline}
\label{sec:outline}

The remaining chapters in this thesis are organised as follows:
\begin{itemize}
\item \textbf{Chapter 2}: We first outline appropriate background information for the reader. Including information pertaining to the source of 
our data set, mathematical notation used throughout this thesis, previous work in this area and our research approach and methodology.
\item \textbf{Chapter 3}: In this chapter we discuss different feature vectors for \emph{User Interactions} and the results of applying these feature vectors 
in comparison with our baselines.
\item \textbf{Chapter 4}: A similar feature vector analysis as above, however the feature vectors we utilise are for \emph{User Preferences}.
\item \textbf{Chapter 5}: In this chapter we discuss results from combining different positive feature vectors into a \emph{Positive Feature Combination}
and propose an ideal feature hybrid based on our analysis.
\item \textbf{Chapter 6}: Finally, we draw the work done throughout this thesis to a conclusion and offer avenues for future work in this area.
\end{itemize}

All chapters combined, this thesis represents a novel approach to exploiting and analysing \emph{User Interactions} and \emph{User Preferences} 
to ascertain which features are most indicative of user likes and present an approach of combining these useful feature components into an effective 
classification paradigm.

%%% Local Variables: 
%%% mode: latex
%%% TeX-master: "thesis"
%%% End: 
