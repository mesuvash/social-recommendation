%!TEX root = document.tex

\label{sec:introduction}

% motivate socical recommendation 
Online social networks such as Facebook record a very rich set of user
preferences  (likes of links, posts, photos, videos), user traits,
interactions and activities (conversation streams, tagging, group memberships,
interests, personal history and demographic data).  In the context of
 recent work on social recommendation~\cite{sorec,ste,lla}, and those on 
information diffusion %in social network analysis
~\cite{Goel2012structure,Romero2011hashtag,Bakshy2012chamber}, 
it is important to know which of these interactions or common traits
are actually reflective of common preferences.

Many recommendation methods for social networks 
use an aggregation approach to representing  
user interactions, such as performing 
low-rank factorization of the social relationships matrix~\cite{sorec} 
using a trust ensemble~\cite{ste},
or collapsing co-preferences and interactions as 
one indication of friend similarity~\cite{Noel2012NOF}.
The point of departure for this work, is the hypothesis that
fine-grained {\em affinities}, such as
different modalities of interaction (e.g. commenting vs tagging) 
and that different items relating to social preference 
(e.g. a university alumni group vs fans of a TV series) 
are of different importance for preference prediction.
Moreover, effective use of this information, i.e., {\em filtering}, 
will lead to better results in social recommendation.

To obtain quantitative answers to this question, 
we have built a Facebook App to collect a set of 
detailed user information and detailed interaction history 
available through the Facebook Graph API. 
We recommend links to each App user from the timeline of their 
friends and non-friends, and we record users' explicit likes 
and dislikes for the links being recommended to them. 
In this work, we explore the role of detailed user preferences 
and interactions for recommendation, called {\em social affinity filtering}.
% Our recent work~\cite{anonymous} designed new objective functions
%for social recommendation that out-perform state-of-the-art, 

In our 4-month interaction trace from 119 app users and their 38,000+ friends, 
we use detailed history of 22 different interaction types on users' timeline, 
along with their preferences on 3000+ groups, 4000+ favourites, and 10,000+ pages. 
%Do fine-grained social interactions help predict user preferences?
We found that social affinity filtering significantly 
outperforms state-of-the-art social recommendation, by up to 6\% in accuracy.
We also found that groups, pages, and favourites, 
as rich and active indications of user preference, outperforms interaction traces. 
Among the interactions, we found that those on videos are more predictive than those on other content types (photos, post, link), and that outgoing interactions (performed by the ego) 
is more predictive than incoming ones (performed by friends on the ego's timeline).
Among {\em groups}, {\em pages} and {\em favourites}, we found that the most  
predictive features have smaller membership size, and that favorite features corresponding to 
persistent activities (such as TV, books, activities) are more predictive than generic or 
temporally synchronized activities (such as interests, sports). 
These findings points to new directions for improving social recommendation, 
and open up interesting questions about the nature of interactions and social preferences. 

%\TODO{add table-of-content?}

%In this paper,
%we provide empirical answers for questions such as the following:
%\begin{itemize}
%%%lexing: make it sound a bit less technical and maintain the specificity, this is icwsm, hem
%%\item What is the probability that 
%\item How likely will a random user $u$ like
%photos that have been liked at least $k$ times by friends? 
%Or friends whose posts $u$ has commented on?
%%\item What is the probability that a random user $u$ will like any
%\item How likely will a random user $u$ like a
%link that is liked at least $k=2$ times by friends sharing common interests
%(movies, television, etc.), or sharing life history (same school or employer)? 
%What if the maximum number of friends sharing this interest is at most $n$?
%%\item What is the probability that a random male user $u$ 
%\item How likely will a male user $u$ like
%items liked at least $k=1$ times by his female friends?
%\end{itemize}

%While many of these queries may seem very narrow, we note that subtle
%changes in the query conditions and parameters above can lead to
%drastic swings in predictive probability.  For example, in the above
%queries, large variations in probability may be 
%observed by changing photo likes to post likes, changing
%wall comment interactions to photo comments, changing from $k=1$ to
%$k=2$ or $n=2$ to $n=4$, and changing the group definition from common
%school to common employer.

\eat{
To better understand these subtleties and to understand what
social interactions and user traits reflect common preferences on
Facebook, we proceed in the following sections to describe our data,
our experimental methodology, and various analyses according to our
methodology that shed light on the above questions. 
On one hand our observations confirm certain observations made previous 
on different networks, such as the diminishing returns of repeated exposures, 
on the other we also see a few new clues such as 
that very specific types of outgoing interactions are more predictive 
than other interactions. 
We then conclude with a summary of the key novel observations arising
from this study.
}
